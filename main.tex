%%%%%%%%%%%%%%%%%%%%%%%%%%%%%%%%%%%%%%%%%%%%%%%%%%%%%%%%%%%%%%%%%%%%%%%%%%%%%%%%
%%%%%%%%%%%%%%%%%%%%%%%%%%%%%%%%%%%%%%%%%%%%%%%%%%%%%%%%%%%%%%%%%%%%%%%%%%%%%%%%
%%                                                                            %%
%% thesistemplate.tex version 3.10 (2018/04/24)                               %%
%% The LaTeX template file to be used with the aaltothesis.sty (version 3.10) %%
%% style file.                                                                %%
%% This package requires pdfx.sty v. 1.5.84 (2017/05/18) or newer.            %%
%%                                                                            %%
%% This is licensed under the terms of the MIT license below.                 %%
%%                                                                            %%
%% Copyright 2017-2018, by Luis R.J. Costa, luis.costa@aalto.fi,              %%
%% Copyright 2017-2018 documentation in Finnish in the template by Perttu     %%
%% Puska, perttu.puska@aalto.fi                                               %%
%% Copyright Swedish translations 2017-2018 by Elisabeth Nyberg,              %%
%% elisabeth.nyberg@aalto.fi and Henrik Wallén, henrik.wallen@aalto.fi        %%
%%                                                                            %%
%% Permission is hereby granted, free of charge, to any person obtaining a    %%
%% copy of this software and associated documentation files (the "Software"), %%
%% to deal in the Software without restriction, including without limitation  %%
%% the rights to use, copy, modify, merge, publish, distribute, sublicense,   %%
%% and/or sell copies of the Software, and to permit persons to whom the      %%
%% Software is furnished to do so, subject to the following conditions:       %%
%% The above copyright notice and this permission notice shall be included in %%
%% all copies or substantial portions of the Software.                        %%
%% THE SOFTWARE IS PROVIDED "AS IS", WITHOUT WARRANTY OF ANY KIND, EXPRESS OR %%
%% IMPLIED, INCLUDING BUT NOT LIMITED TO THE WARRANTIES OF MERCHANTABILITY,   %%
%% FITNESS FOR A PARTICULAR PURPOSE AND NONINFRINGEMENT. IN NO EVENT SHALL    %%
%% THE AUTHORS OR COPYRIGHT HOLDERS BE LIABLE FOR ANY CLAIM, DAMAGES OR OTHER %%
%% LIABILITY, WHETHER IN AN ACTION OF CONTRACT, TORT OR OTHERWISE, ARISING    %%
%% FROM, OUT OF OR IN CONNECTION WITH THE SOFTWARE OR THE USE OR OTHER        %%
%% DEALINGS IN THE SOFTWARE.                                                  %%
%%                                                                            %%
%%                                                                            %%
%%%%%%%%%%%%%%%%%%%%%%%%%%%%%%%%%%%%%%%%%%%%%%%%%%%%%%%%%%%%%%%%%%%%%%%%%%%%%%%%
%%                                                                            %%
%%                                                                            %%
%% An example for writting your thesis using LaTeX                            %%
%% Original version and development work by Luis Costa, changes by Perttu     %% 
%% Puska.                                                                     %%
%% Support for Swedish added 15092014                                         %%
%% PDF/A-b support added on 15092017                                          %%
%% PDF/A-2b and PDF/A-3b support added on 24042018                            %%
%%                                                                            %%
%% This example consists of the files                                         %%
%%         thesistemplate.tex (versio 3.10)                                   %%
%%         opinnaytepohja.tex (versio 3.10) (for text in Finnish)             %%
%%         aaltothesis.cls (versio 3.10)                                      %%
%%         kuva1.eps (graphics file)                                          %%
%%         kuva2.eps (graphics file)                                          %%
%%         kuva1.jpg (graphics file)                                          %%
%%         kuva2.jpg (graphics file)                                          %%
%%         kuva1.png (graphics file)                                          %%
%%         kuva2.png (graphics file)                                          %%
%%         kuva1.pdf (graphics file)                                          %%
%%         kuva2.pdf (graphics file)                                          %%
%%                                                                            %%
%%                                                                            %%
%% Typeset in Linux either with                                               %%
%% pdflatex: (recommended method)                                             %%
%%             $ pdflatex thesistemplate                                      %%
%%             $ pdflatex thesistemplate                                      %%
%%                                                                            %%
%%   The result is the file thesistemplate.pdf that is PDF/A compliant, if    %%
%%   you have chosen the proper \documenclass options (see comments below)    %%
%%   and your included graphics files have no problems.
%%                                                                            %%
%% Or                                                                         %%
%% latex:                                                                     %%
%%             $ latex thesistemplate                                         %%
%%             $ latex thesistemplate                                         %%
%%                                                                            %%
%%   The result is the file thesistemplate.dvi, which is converted to ps      %%
%%   format as follows:                                                       %%
%%                                                                            %%
%%             $ dvips thesistemplate -o                                      %%
%%                                                                            %%
%%   and then to pdf as follows:                                              %%
%%                                                                            %%
%%             $ ps2pdf thesistemplate.ps                                     %%
%%                                                                            %%
%%   This pdf file is not PDF/A compliant. You must must make it so using,    %%
%%   e.g., Acrobat Pro or PDF-XChange.                                        %%
%%                                                                            %%
%%                                                                            %%
%% Explanatory comments in this example begin with the characters %%, and     %%
%% changes that the user can make with the character %                        %%
%%                                                                            %%
%%%%%%%%%%%%%%%%%%%%%%%%%%%%%%%%%%%%%%%%%%%%%%%%%%%%%%%%%%%%%%%%%%%%%%%%%%%%%%%%
%%%%%%%%%%%%%%%%%%%%%%%%%%%%%%%%%%%%%%%%%%%%%%%%%%%%%%%%%%%%%%%%%%%%%%%%%%%%%%%%
%%
%% WHAT is PDF/A
%%
%% PDF/A is the ISO-standardized version of the pdf. The standard's goal is to
%% ensure that he file is reproducable even after a long time. PDF/A differs
%% from pdf in that it allows only those pdf features that support long-term
%% archiving of a file. For example, PDF/A requires that all used fonts are
%% embedded in the file, whereas a normal pdf can contain only a link to the
%% fonts in the system of the reader of the file. PDF/A also requires, among
%% other things, data on colour definition and the encryption used.
%% Currently three PDF/A standards exist:
%% PDF/A-1: based on PDF 1.4, standard ISO19005-1, published in 2005.
%%          Includes all the requirements essential for long-term archiving.
%% PDF/A-2: based on PDF 1.7, standard ISO19005-2, published in 2011.
%%          In addition to the above, it supports embedding of OpenType fonts,
%%          transparency in the colour definition and digital signatures.
%% PDF/A-3: based on PDF 1.7, standard ISO19005-3, published in 2012.
%%          Differs from the above only in that it allows embedding of files in
%%          any format (e.g., xml, csv, cad, spreadsheet or wordprocessing
%%          formats) into the pdf file.
%% PDF/A-1 files are not necessarily PDF/A-2 -compatible and PDF/A-2 are not
%% necessarily PDF/A-1 -compatible.
%% All of the above PDF/A standards have two levels:
%% b: (basic) requires that the visual appearance of the document is reliably
%%    reproduceable.
%% a (accessible) in addition to the b-level requirements, specifies how
%%   accessible the pdf file is to assistive software, say, for the physically
%%   impaired.
%% For more details on PDF/A, see, e.g., https://en.wikipedia.org/wiki/PDF/A
%%
%%
%% WHICH PDF/A standard should my thesis conform to?
%%
%% Primarily to the PDF/A-1b standard. All the figures and graphs typically
%% use in thesis work do not require transparency features, a basic '2-D'
%% visualisation suffices. The font to be used are specified in this template
%% and they should not be changed. However, if you have figures where
%% transparency characteristics matter, use the PDF/A-2b standard. Do not use
%% the PDF/A-3b standard for your thesis.
%%
%%
%% WHAT graphics format can I use to produce my PDF/A compliant file?
%%
%% When using pdflatex to compile your work, use jpg, png or pdf files. You may
%% have PDF/A compliance problems with figures in pdf format. Do not use PDF/A
%% compliant graphics files.
%% If you decide to use latex to compile your work, the only acceptable file
%% format for your figure is eps. DO NOT use the ps format for your figures.

%% USE one of these:
%% * the first when using pdflatex, which directly typesets your document in the
%%   chosen pdf/a format and you want to publish your thesis online,

%% * the second when you want to print your thesis to bind it, or
%% * the third when producing a ps file and a pdf/a from it.
%%
\documentclass[english, 12pt, a4paper, elec, utf8, a-1b, online]{aaltothesis}
%\documentclass[english, 12pt, a4paper, elec, utf8, a-1b]{aaltothesis}
%\documentclass[english, 12pt, a4paper, elec, dvips, online]{aaltothesis}

%% Use the following options in the \documentclass macro above:
%% your school: arts, biz, chem, elec, eng, sci
%% the character encoding scheme used by your editor: utf8, latin1
%% thesis language: english, finnish, swedish
%% make an archiveable PDF/A-1b, PDF/A-2b or PDF/A-3b compliant file: a-1b,
%%                    a-2b, a-3b
%%                    (a normal pdf is produced without the a-*b option)
%% typeset in symmetric layout and blue hypertext for online publication: online
%%            (no option is the default, resulting in a wide margin on the
%%             binding side of the page and black hypertext)
%% two-sided printing: twoside (default is one-sided printing)
%%

%% Use one of these if you write in Finnish (see the Finnish template
%% opinnaytepohja.tex)
%\documentclass[finnish, 12pt, a4paper, elec, utf8, a-1b, online]{aaltothesis}
%\documentclass[finnish, 12pt, a4paper, elec, utf8, a-1b]{aaltothesis}
%\documentclass[finnish, 12pt, a4paper, elec, dvips, online]{aaltothesis}

\usepackage{graphicx}

%% Math fonts, symbols, and formatting; these are usually needed
\usepackage{amsfonts,amssymb,amsbsy}

%% Change the school field to specify your school if the automatically set name
%% is wrong
% \university{aalto-yliopisto}
% \school{Sähkötekniikan korkeakoulu}

%% Edit to conform to your degree programme
%%
\degreeprogram{Electronics and electrical engineering}
%%

%% Your major
%%
\major{Communications Engineering}
%%

%% Major subject code
%%
\code{ELEC0007}
%%
 
%% Choose one of the three below
%%
%\univdegree{BSc}
\univdegree{MSc}
%\univdegree{Lic}
%%

%% Your name (self explanatory...)
%%
\thesisauthor{Rohan Krishnakumar}
%%

%% Your thesis title comes here and possibly again together with the Finnish or
%% Swedish abstract. Do not hyphenate the title, and avoid writing too long a
%% title. Should LaTeX typeset a long title unsatisfactorily, you mght have to
%% force a linebreak using the \\ control characters.
%% In this case...
%% Remember, the title should not be hyphenated!
%% A possible "and" in the title should not be the last word in the line, it
%% begins the next line.
%% Specify the title again without the linebreak characters in the optional
%% argument in box brackets. This is done because the title is part of the 
%% metadata in the pdf/a file, and the metadata cannot contain linebreaks.
%%
\thesistitle{Accelerated DPDK in containers for networking nodes}
%\thesistitle[Title of the thesis]{Title of\\ the thesis}
%%

%%
\place{Espoo}
%%

%% The date for the bachelor's thesis is the day it is presented
%%
\date{24.4.2018}
%%

%% Thesis supervisor
%% Note the "\" character in the title after the period and before the space
%% and the following character string.
%% This is because the period is not the end of a sentence after which a
%% slightly longer space follows, but what is desired is a regular interword
%% space.
%%
\supervisor{Prof.\ Yu Xiao}
%%
%% Advisor(s)---two at the most---of the thesis.
%%
\advisor{D.Sc.(tech) Vesa Hirvisalo}
\advisor{MSc.\ Kati Ilvonen}
%\advisor{MSc Sarah Scientist}
%%

%% Aaltologo: syntax:
%% \uselogo{aaltoRed|aaltoBlue|aaltoYellow|aaltoGray|aaltoGrayScale}{?|!|''}
%% The logo language is set to be the same as the thesis language.
%%
\uselogo{aaltoRed}{''}
%%

%% The English abstract:
%% All the details (name, title, etc.) on the abstract page appear as specified
%% above.
%% Thesis keywords:
%% Note! The keywords are separated using the \spc macro
%%
\keywords{NFV\spc DPDK \spc Containers\spc OvS\spc SR-IOV\spc Kubernetes \spc CNI}
%%

%% The abstract text. This text is included in the metadata of the pdf file as well
%% as the abstract page.
%%
\thesisabstract{
Your abstract in English. Keep the abstract short. The abstract explains your 
research topic, the methods you have used, and the results you obtained. In the 
PDF/A format of this thesis, in addition to the abstract page, the abstract text is 
written into the pdf file's metadata. Write here the text that goes into the 
metadata. The metadata cannot contain special characters, linebreak or paragraph 
break characters, so these must not be used here. If your abstract does not contain 
special characters and it does not require paragraphs, you may take advantage of 
the abstracttext macro (see the comment below). Otherwise, the metadata abstract 
text must be identical to the text on the abstract page.
}

%% Copyright text. Copyright of a work is with the creator/author of the work
%% regardless of whether the copyright mark is explicitly in the work or not.
%% You may, if you wish, publish your work under a Creative Commons license (see
%% creaticecommons.org), in which case the license text must be visible in the
%% work. Write here the copyright text you want. It is written into the metadata
%% of the pdf file as well.
%% Syntax:
%% \copyrigthtext{metadata text}{text visible on the page}
%% 
%% In the macro below, the text written in the metadata must have a \noexpand
%% macro before the \copyright special character, and macros (\copyright and
%% \year here) must be separated by the \ character (space chacter) from the
%% text that follows. The macros in the argument of the \copyrighttext macro
%% automatically insert the year and the author's name. (Note! \ThesisAuthor is
%% an internal macro of the aaltothesis.cls class file).
%% Of course, the same text could have simply been written as
%% \copyrighttext{Copyright \noexpand\copyright\ 2018 Eddie Engineer}
%% {Copyright \copyright{} 2018 Eddie Engineer}
%%
\copyrighttext{Copyright \noexpand\copyright\ \number\year\ \ThesisAuthor}
{Copyright \copyright{} \number\year{} \ThesisAuthor}

%% You can prevent LaTeX from writing into the xmpdata file (it contains all the 
%% metadata to be written into the pdf file) by setting the writexmpdata switch
%% to 'false'. This allows you to write the metadata in the correct format
%% directly into the file thesistemplate.xmpdata.
%\setboolean{writexmpdatafile}{false}

%% All that is printed on paper starts here
%%
\begin{document}

%% Create the coverpage
%%
\makecoverpage

%% Typeset the copyright text.
%% If you wish, you may leave out the copyright text from the human-readable
%% page of the pdf file. This may seem like a attractive idea for the printed
%% document especially if "Copyright (c) yyyy Eddie Engineer" is the only text
%% on the page. However, the recommendation is to print this copyright text.
%%
\makecopyrightpage

%% Note that when writting your thesis in English, place the English abstract
%% first followed by the possible Finnish or Swedish abstract.

%% Abstract text
%% All the details (name, title, etc.) on the abstract page appear as specified
%% above.
%%
\begin{abstractpage}[english]
  Your abstract in English. Keep the abstract short. The abstract explains your
  research topic, the methods you have used, and the results you obtained.  
  
  The abstract text of this thesis is written on the readable abstract page as
  well as into the pdf file's metadata via the $\backslash$thesisabstract macro
  (see above). Write here the text that goes onto the readable abstract page.
  You can have special characters, linebreaks, and paragraphs here. Otherwise,
  this abstract text must be identical to the metadata abstract text.
  
  If your abstract does not contain special characters and it does not require
  paragraphs, you may take advantage of the abstracttext macro (see the comment
  below).
\end{abstractpage}

%% The text in the \thesisabstract macro is stored in the macro \abstractext, so
%% you can use the text metadata abstract directly as follows:
%%
%\begin{abstractpage}[english]
%	\abstracttext{}
%\end{abstractpage}

%% Preface
%%
\mysection{Preface}
I want to thank Professor Pirjo Professori and my instructor Dr Alan Advisor for 
their good and poor guidance.\\

\vspace{5cm}
Otaniemi, 24.4.2018

\vspace{5mm}
{\hfill Rohan Krishnakumar \hspace{1cm}}

%% Force a new page after the preface
%%
\newpage


%% Table of contents. 
%%
\thesistableofcontents


%% Symbols and abbreviations
\mysection{Abbreviations}

\subsection*{Symbols}

\begin{tabular}{ll}

$\mathbf{B}$  & magnetic flux density  \\
$c$              & speed of light in vacuum $\approx 3\times10^8$ [m/s]\\
$\omega_{\mathrm{D}}$    & Debye frequency \\
$\omega_{\mathrm{latt}}$ & average phonon frequency of lattice \\
$\uparrow$       & electron spin direction up\\
$\downarrow$     & electron spin direction down
\end{tabular}

\subsection*{Operators}

\begin{tabular}{ll}
$\nabla \times \mathbf{A}$              & curl of vectorin $\mathbf{A}$\\
$\displaystyle\frac{\mbox{d}}{\mbox{d} t}$ & derivative with respect to 
variable $t$\\[3mm]
$\displaystyle\frac{\partial}{\partial t}$  & partial derivative with respect 
to variable $t$ \\[3mm]
$\sum_i $                       & sum over index $i$\\
$\mathbf{A} \cdot \mathbf{B}$    & dot product of vectors $\mathbf{A}$ and 
$\mathbf{B}$
\end{tabular}

\subsection*{Abbreviations}

\begin{tabular}{ll}
DPDK        & Data Plane Development Kit \\
NFV         & Network Function Virtualization \\
SR-IOV      & Single Root IO Vitrualization \\
k8s         & Kubernetes \\
\end{tabular}


%% \clearpage is similar to \newpage, but it also flushes the floats (figures
%% and tables).
%%
\cleardoublepage

%% Text body begins. Note that since the text body is mostly in Finnish the
%% majority of comments are also in Finnish after this point. There is no point
%% in explaining Finnish-language specific thesis conventions in English.
%% This text will be translated to English soon.
%%
\section{Introduction}

%% Leave page number of the first page empty
%% 
\thispagestyle{empty}
In telecommunication industry, the network node functions have traditionally run on operator proprietary hardware which are dedicated for a specific purpose, known as ASICs (Application Specific Integrated Circuits). This has led to long development and deployment cycles, higher cost of operation, etc., thereby hindering rapid technological advancement in the field. Slowly the network node functions are being replaced by virtualized software solutions that can run on generic computing hardware; thereby addressing the above problems and making the network more flexible. This network architecture where virtualized software solutions are chained together or combined to form network node functions is called Network Function Virtualization (NFV).

In the NFV nomenclature, the software implementation of the network node functions or sub-functions are referred to as Virtual Network Functions (VNF). For instance, a network firewall, a DHCP server or a NAT function. The current de-facto VNF implementations run on virtual machines such as a Kernel-based Virtual Machine (KVM). But with the recent advent of containers and their gaining popularity due to some of their obvious advantages over virtual machines, there is a need to explore their suitability in NFV. Containers are comparatively light weight and utilize the same host kernel when compared to virtual machines. They can be booted up with minimum latency [] which implies, container based deployments focus on on-demand spawning of the containerized application. For example, a NAT application running on a container could be spawned up based on the number of clients utilizing it, with one container handling one or a fixed number of clients. However, utilizing containers in NFV is still in its nascent stages and there are quite a few issues to be explored and addressed [][].
%% [ https://www.redhat.com/blog/verticalindustries/is-nfv-ready-for-containers/ ], there are several on going studies and  containerization in NFV is still in a nascent stage and  As listed in [ https://www.ietf.org/proceedings/94/slides/slides-94-nfvrg-11.pdf ], some of the key features needed in NFV environment are performance, portability/scalability and isolation. 

The scope of the thesis is to focus on the networking aspect of utilizing containers in NFV. An import requirement in NFV is fast packet processing when compared to normal container based deployments. The thesis explores container networking both in a standalone environment and also in a container orchestrated environment, utilizing Data Plane Development Kit (DPDK) for fast packet processing and Single Root IO Virtualization (SR-IOV) for hardware virtualization.

A container orchestrated environment could be visualized as a cloud of multiple servers with multiple containers spawned on request and killed after serving the request. A fast data path in this scenario would include DPDK supported interfaces that are exposed to the external network and also a fast packet forwarding within the cloud to reach the container of interest. When the orchestrator deploys a container, there are constraints to make sure that the fast data path is exposed to it. Such a networking requirement could be critical in NFV and finding a suitable generic solution to it is one of the key questions driving this thesis.

DPDK utilizes several techniques for better packet processing such as skipping the kernel network stack and running applications in userspace, using poll mode drivers instead interrupt driven ones, hugepages, Direct Memory Access (DMA) and others. Among them correct utilization of hugepages is an important aspect especially when multiple DPDK based applications are run simultaneously in the same host. Most of the current container run-time applications such as Docker,.. do not support hugepage isolation. It is possible that one application ends up consuming or pre-allocating all the hugepages. Addressing this drawback to support hugepage isolation in Docker is another key aspect of this thesis.

Whenever possible, for a better grasp of the performance measurements with DPDK/SR-IOV, a comparative study with and without them is done. For some of the basic tools, only the industry de-facto solutions are considered, such as  Docker for container run-time and Kubernetes for container orchestration. Keeping these constant, the exploration is on the various networking solutions feasible for them and other aspects described above.

%%While carrying out the experiments, scalability and generic nature of the solution have been given important consideration.


%%Some questions that need answering are : 
	%%- Hugepage isolation in containers, freeing of hugepage (on failure of DPDK application?) . It is important to consider multiple DPDK based apps running on the same host [2]
	%%- The same plug-in uses DPDK on VF on the host and just exposes it to container
	
%%Whenever any decision on technologies to be used is taken, it is done so with NFV consideration. Also, for container orchestration, Kubernetes is de-facto solution and is the only considered one in the thesis.

The next section in the thesis covers the related technologies utilized, their definitions and how they work. The motivation for them and in some cases their evolution. The third chapter is a prelude to the experimentation part which is chapter four.

\clearpage
\section{How stuffs work}
This section describes the main technologies, tools and other components used in the thesis and how they work.
\subsection{Data Plane Development Kit, DPDK}
DPDK is a "set of libraries and drivers"[1] that enable fast packet processing. It is an open source software framework that can be used to build fast networking applications and is maintained by the Linux Foundation[2]. As compared to a normal kernel network stack, DPDK can provide about twenty five percent [3] improvement in packet processing speed.

The driving factor behind the development of DPDK is to address the handling of extremely fast packet rates, especially in a communication network infrastructure, where the packet sizes are typically smaller and packet rates much higher. The average CPU cycles available for handling a packet is much less. For instance, a 10 Gigabit Ethernet card receiving packets of size 1024 bytes would receive 1.25 million packets per second. A CPU with 2GHz clock cycles handling these packets will have 1600 cycles per packet. For a packet size of 64 bytes, this would be 19.5 million packets per second and 102 cycles per packet. 

DPDK utilizes a verity of techniques to address improve the packet processing. Primarily, it uses Poll Mode Drivers (PMD) instead of interrupt driver drivers to handle packets. In traditional packet handling, when a packet is received, the CPU does a context switching from the userspace process to kernel, runs the Interrupt Service Routine (ISR) and switches back to userspace. This is a big overhead especially for higher packet rates. Poll Mode Drivers run on the userspace and use a dedicated CPU core to handle the traffic on one or multiple interfaces. DPDK uses pthread affinity to disable the kernel scheduler from utilizing those cores.

DPDK utilizes hugepages. Hugepages simply imply bigger page size. The standard Linux page size is 4kB while hugepage size could be 2MB or 1GB. In a normal memory access, the virtual memory address needs to be translated to physical memory before it can be accessed. The critical factor affecting performance is the TLB hit. When bigger page sizes are used, there are fewer pages and higher probability of TLB hit.

Software prefetching in addition to H/W prefetching. 

Intel DDIO : Available only in some supported network cards. Network packets are directly put into L3 cache instead of being DMA'd into main memory.

Cache alignment : Intel cache lines are 64bytes. Cache alignment increases performance. If data fetched from memory is put in different cache lines, it would require multiple cache fetches to retrieve the data. Alignment, putting in one cache line, decreases cache reads.

DPDK uses Direct Memory Access to speed up packet buffer copying. Compared to a normal kernel network
stack, when a packet is received, DPDK uses DMA to directly lift the buffer to user space. In traditional packet handling in the kernel stack involves copying of buffer multiple times such as, from the NIC buffer to kernel socket buffer (skbuf in Linux) and from there to the userspace. This reduces performance due to the copying overhead and also due to loss of localization of data leading to fewer cache hits. The method used in DPDK called zero-copy.

Most of the DPDK components run on the userspace avoiding context switching altogether. The packets received on an interface are DMA'd directly to the poll mode drivers. 

Modes of operation of a DPDK application - Run to completion. Each core is assigned a port or a set of ports. The core does the I/O specifically for that port. It accept the incoming packet, process it and send it out through the same port or set of ports. Pipeline model - One core is dedicated to just I/O. All packets received at this port and then sent to different cores using ring buffers. It is more complex and could result in packet order being not maintained.

%% [1] “DPDK, Data Plane Development Kit”, Retrieved from dpdk.org
%% [2] “Linux Foundation Projects”, Retrieved from https://www.linuxfoundation.org/projects/
%% [3] https://builders.intel.com/university/networkbuilders/course/dpdk-101

The main components of DPDK are the core libraries, Poll Mode Drivers for various supported Network Interface Cards (NICs) and other libraries that deal with packet classification, Quality of Service (QoS) and specific to the different platforms [4]. All of these libraries run on the userpace. Also, apart from theses, DPDK uses PCI drivers in the kernel such as IGB-UIO or VFIO-PCI that do basic functionality. When an interface is bound to DPDK, it is detached from the normal kernel driver and attached to one of these DPDK drivers in the kernel.

%% http://doc.dpdk.org/guides-16.04/prog_guide/overview.html

The core libraries in DPDK has an Environment Abstraction Layer (EAL) which hides or abstracts the platform from the libraries and applications running above it. This also deals with memory allocation in huge pages and PCI related buffer handling. DPDK does not allocate memory at run-time but rather pre-allocates memory during initialization. For packet processing, use memory from pool. Return after use. The libraries related to allocating pool of memory (Mempool) consisting of memory buffer (Mbuf) and lockless queues or ring buffers for handling them are part of the core library.

\subsection{Single Root IO Virtualization, SR-IOV}
SR-IOV is a virtualization of a PCI Express (PCIe) device, or in other words, a NIC, into multiple virtual devices that can be directly assigned to an instance of a virtual machine or container. It is hardware virtualization with SR-IOV has a Physical Function (PF) which acts like a normal PCIe device and multiple Virtual Functions (VF) that act like lightweight PCIe devices. In a virtual machine context, the host machine kernel can be bypassed and the packets can flow directly between the VF and PF.

Consider the scenario where one host is running multiple Virtual Machines (VMs) and a NIC receiving and transmitting packets for all those VMs. When a packet is received by the NIC, it triggers an interrupt to the CPU core that is assigned for handling NIC interrupts. This core services the request and exmines the packet. Based on the MAC or VLAN, it triggers another interrupt on the core that is servicing the virtual machine. This is an overhead since there are multiple interrupt handling in the host even before the packet is transferred to the guest operating system. Moreover, when the packet rate is high, one core handling all the packets for all the VMs can be a severe bottle neck. As an enhancement to this, Intel 
\subsection{Container}
\subsection{Kubernetes}
Kubernetes is container orchestration. It could be 
\subsection{Open vSwitch}
\section{Background}
	History, theory, current practices
	Definitions
		CNI
		SR-IOV
		OvS

	Considerations
		Packet generators
%%
%% Three levels of hierarchy in sectioning should be enough

\subsection*{Containers}
\subsection*{Container orchestration, Kubernetes}
\subsection*{Data Plane Development Kit}
\subsection*{Single Root IO Virtualization}

\subsection*{Open vSwitch}
\clearpage
\section{Research material and methods}

\clearpage
\section{Core - Implementation, experiments}
\subsection{First step - Multiple interfaces}
Experiment with SR-IOV VF exposed to container and DPDK testpmd running on it.
\subsection{Building a Kubernetes cluster}
Using MAAS for building a cluster from bare metal.
Two approaches to deploying k8s - juju and kubeadm
\subsection{Kubernetes with multus}
Intel Node Feature Discovery
\subsection{OvS-DPDK and container}
\subsection{Kubernetes with OvN}
Exposing external interface and routing within
\clearpage
\section{Results}

\clearpage
\section{Summary} 

\clearpage
\thesisbibliography
\begin{thebibliography}{99}

%% Alla pilkun j\"alkeen on pakotettu oikea v\"ali \<v\"alily\"onti>-merkeill\"a.
\bibitem{Kauranen} Kauranen,\ I., Mustakallio,\ M. ja Palmgren,\ V.
  \textit{Tutkimusraportin kirjoittamisen opas opinn\"aytety\"on
    tekij\"oille.}  Espoo, Teknillinen korkeakoulu, 2006.

\bibitem{Itkonen} Itkonen,\ M. \textit{Typografian k\"asikirja.} 3.\
  painos.  Helsinki, RPS-yhti\"ot, 2007.

\bibitem{Koblitz} Koblitz,\ N. \textit{A Course in Number Theory and
    Cryptography. Graduate Texts in Mathematics 114.}  2.\ painos. New
  York, Springer, 1994.

%% Kun on useampi nimikirjain, jokaisen nimikirjaimen v\"aliin
%% kuuluu v\"alily\"onti. Oikea v\"alin m\"a\"ar\"a on saatu \<v\"alily\"onnill\"a>
\bibitem{bcs} Bardeen,\ J., Cooper,\ L.\ N. ja Schrieffer,\ J.\ R.
  Theory of Superconductivity. \textit{Physical Review,} 1957, vol.\
  108, nro~5, s.\ 1175--1204.

\bibitem{Deschamps} Deschamps,\ G.\ A. Electromagnetics and
  Differential Forms. \textit{Proceedings of the IEEE,} 1981, vol.\
  69, nro~6, s.\ 676--696.

%% Alla esimerkki englanninkielisen tavuttamisen pakottamisesta.
%% Oletusarvoisesti k\"aytet\"a\"an suomalaista tavutusta, mutta viitteiss\"a
%% esiintyy usein muunkielisi\"a lauseita, jotka tulevat siten tavutetuksi
%% suomen kielen s\"a\"ant\"ojen mukaan. T\"am\"an voi korjata \foreignlanguage-
%% komennolla, jonka ensimm\"ainen parametri on vieraan kielen nimi ja toinen 
%% on vieraalla kielell\"a tavutettava teksti. 
\bibitem{Sihvola} Sihvola,\ A.\ et al.
  \foreignlanguage{english}{Interpretation of measurements of helix 
    and bihelix superchiral structures.}
  Teoksessa: Jacob,\ A.\ F. ja
  Reinert,\ J. (toim.) \textit{Bianisotropics '98 7th International
    Conference on Complex Media.}  Braunschweig, 3.--6.6.1998.
  Braunscweig, Technische Universit\"at Braunschweig, 1998, s.\
  317--320.

%% Alla on suomalainen yhdistelm\"asukunimi. Sen nimien v\"aliss\"a 
%% k\"aytet\"a\"an yhdysmerkki\"a l. tavuviivaa, kirjoitetaan -.
\bibitem{Lindblom} Lindblom-Yl\"anne,\ S. ja Wager,\ M.  Tieteellisten
  opinn\"aytet\"oiden ohjaaminen. Teoksessa: Lindblom-Yl\"anne,\ S. ja
  Nevgi,\ A. (toim.) \textit{Yliopisto- ja korkeakouluopettajan
    k\"asikirja.}  Helsinki, WSOY, 2004, s.\ 314--325.
 
\bibitem{Miinusmaa} Miinusmaa,\ H. Neliskulmaisen rei\"an poraamisesta
  kolmikulmaisella poralla. Diplomity\"o, Teknillinen korkeakoulu,
  konetekniikan osasto, Espoo, 1977.

%% T\"ass\"a taas pakotettu englanninkielinen tavutus. 
%% Pedanttinen kirjoittaja pakottaa tietysti jokaiseen englanninkieliseen
%% lauseeseen englannin tavutuksen, mutta t\"ass\"a esityksess\"a ei n\"ain ole
%% tehty selvyyden ja l\"ahdekoodin luettavuuden takia. 
\bibitem{Loh} Loh,\ N.\ C. High-Resolution Micromachined
  Interferometric Accelerometer. Master's Thesis, Massachusetts
  Institute of Technology, Cambridge,
  \foreignlanguage{english}{Massachusetts,} 1992.

\bibitem{Lonnqvist} L\"onnqvist,\ A.
  \foreignlanguage{english}{Applications of hologram-based compact
    range: antenna radiation pattern, radar cross section, and
    absorber reflectivity measurements.}
  V\"ait\"oskirja, Teknillinen korkeakoulu, s\"ahk\"o- ja tietoliikennetekniikan
  osasto, 2006.

\bibitem{sfs} SFS 5342. Kirjallisuusviitteiden laatiminen. 2.\ painos.
  Helsinki, Suomen standardisoimisliitto, 2004. 20~s.

\bibitem{haastattelu} Palmgren,\ V. Suunnittelija. Teknillinen
  korkeakoulu, kirjasto. Otaniementie 9, 02150 Espoo. Haastattelu
  15.1.2007.

\bibitem{Ribeiro} Ribeiro,\ C.\ B., Ollila,\ E. ja Koivunen,\ V.
  \foreignlanguage{english}{Stochastic Maximum-Likelihood Method for
    MIMO Propagation Parameter Estimation.}
 \textit{IEEE Transactions
    on Signal Processing,} verkkolehti, vol.\ 55, nro~1, s.\ 46--55.
  Viitattu 19.1.2007. Lehti ilmestyy my\"os painettuna. DOI:
  10.1109/TSP.2006.882057.

\bibitem{Stieber} Stieber,\ T. GnuPG Hacks. \textit{Linux Journal,}
  verkkolehti, 2006, maaliskuu, nro~143. Viitattu 19.1.2007. Lehti
  ilmestyy my\"os painettuna. Saatavissa:
  \url{http://www.linuxjournal.com/article/8732.}

\bibitem{kone} Pohjois-Koivisto,\ T. Voiko kone tulevaisuudessa arvata
  tahtosi?  \textit{Apropos,} verkkolehti, helmikuu, nro~1, 2005.
  Viitattu 19.1.2007.  Saatavissa:
  \url{http://www.apropos.fi/1-2005/prima.php.}

\bibitem{Adida} Adida,\ B.  Advances in Cryptographic Voting Systems.
  Verkkodokumentti. Ph.D.\ Thesis, Massachusetts Institute of
  Technology, Cambridge, 
  \foreignlanguage{english}{Massachusetts,}
  2006. Viitattu 19.1.2007.  Saatavissa:
  \url{http://crypto.csail.mit.edu/~cis/theses/adida-phd.pdf.}

\bibitem{viittaaminen} Kilpel\"ainen,\ P. WWW-l\"ahteisiin viittaaminen
  tutkielmatekstiss\"a. Verkkodokumentti. P\"aivitetty 26.11.2001.
  Viitattu 19.1.2007. Saatavissa:
  \url{http://www.cs.uku.fi/~kilpelai/wwwlahteet.html.}

\end{thebibliography}

%% Appendices
%% If you don't have appendices, remove \clearpage and \thesisappendix below.
\clearpage

\thesisappendix

\section{Esimerkki liitteest\"a\label{LiiteA}}

Liitteet eiv\"at ole opinn\"aytteen kannalta v\"altt\"am\"att\"omi\"a ja 
opinn\"aytteen tekij\"an on 
kirjoittamaan ryhtyess\"a\"an hyv\"a ajatella p\"arj\"a\"av\"ans\"a ilman liitteit\"a.
Kokemattomat kirjoittajat, jotka ovat huolissaan
tekstiosan pituudesta, paisuttavat turhan 
helposti liitteit\"a pit\"a\"akseen tekstiosan pituuden annetuissa rajoissa.
T\"all\"a tavalla ei synny hyv\"a\"a opinn\"aytett\"a.   

Liite on itsen\"ainen kokonaisuus, vaikka se t\"aydent\"a\"akin tekstiosaa.
Liite ei siten ole pelkk\"a listaus, kuva tai taulukko, vaan 
liitteess\"a selitet\"a\"an aina sis\"all\"on laatu ja tarkoitus. 

Liitteeseen voi laittaa esimerkiksi listauksia. Alla on 
listausesimerkki t\"am\"an liitteen luomisesta. 

%% Verbatim-ymp\"arist\"o ei muotoile tai tavuta teksti\"a. Fontti on monospace.
%% Verbatim-ymp\"arist\"on sis\"all\"a annettuja komentoja ei LaTeX k\"asittele. 
%% Vasta \end{verbatim}-komennon j\"alkeen jatketaan k\"asittely\"a.
\begin{verbatim}
	\clearpage
	\appendix
	\addcontentsline{toc}{section}{Liite A}
	\section*{Liite A}
	...
	\thispagestyle{empty}
	...
	teksti\"a
	...
	\clearpage
\end{verbatim}

Kaavojen numerointi muodostaa liitteiss\"a oman kokonaisuutensa:
\begin{eqnarray}
d \wedge A  &=& F, \label{liitekaava1}\\
d \wedge F  &=& 0. \label{liitekaava2}
\end{eqnarray}


\clearpage
\section{Toinen esimerkki liitteest\"a\label{LiiteB}}

%% Liitteiden kaavat, taulukot ja kuvat numeroidaan omana kokonaisuutenaan
%%
%% Equations, tables and figures have their own numbering in Appendices
%\renewcommand{\theequation}{B\arabic{equation}}
%\setcounter{equation}{0}  
%\renewcommand{\thefigure}{B\arabic{figure}}
%\setcounter{figure}{0}
%\renewcommand{\thetable}{B\arabic{table}}
%\setcounter{table}{0}

Liitteiss\"a voi my\"os olla kuvia, jotka
eiv\"at sovi leip\"atekstin joukkoon:
%% Ymp\"arist\"on figure parametrit htb pakottavat
%% kuvan t\"ah\"an, eik\"a LaTeX yrit\"a siirrell\"a niit\"a
%% hyv\"aksi katsomaansa paikkaan. 
%% Ymp\"arist\"o\"a center voi k\"aytt\"a\"a \centering-
%% komennon sijaan
%%
%% Example of a figure, note the use of htb parameters which force
%% the figure to be inserted here
\begin{figure}[htb]
\begin{center}
\includegraphics[height=8cm]{kuva2.pdf}
%\includegraphics[height=8cm]{kuva2.jpg}
%\includegraphics[height=8cm]{kuva2.png}
%\includegraphics[height=8cm]{kuva2.eps}
\end{center}
\caption{Kuvateksti, jossa on liitteen numerointi}
\label{liitekuva}
\end{figure}
%%
Liitteiden taulukoiden numerointi on kuvien ja kaavojen kaltainen:
\begin{table}[htb]
\caption{Taulukon kuvateksti.}
\label{liitetaulukko}
\begin{center}
\fbox{
\begin{tabular}{lp{0.5\linewidth}}
9.00--9.55  & K\"aytett\"avyystestauksen tiedotustilaisuus (osanottajat
ovat saaneet s\"ahk\"opostitse valmistautumisteht\"av\"at, joten tiedotustilaisuus
voidaan pit\"a\"a lyhyen\"a).\\
9.55--10.00 & Testausalueelle siirtyminen
\end{tabular}}
\end{center}
\end{table}
Kaavojen numerointi muodostaa liitteiss\"a oman kokonaisuutensa:
\begin{eqnarray}
T_{ik} &=& -p g_{ik} + w u_i u_k + \tau_{ik},  \label{liitekaava3} \\
n_i    &=& n u_i + v_i.                      \label{liitekaava4}
\end{eqnarray}

\end{document}